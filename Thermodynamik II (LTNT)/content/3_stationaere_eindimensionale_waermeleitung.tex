%!TEX root = ../Thermodynamik II (LTNT).tex

\section{Stationäre eindimensionale Wärmeleitung} % (fold)
	\subsection{Wärmefluss durch die ebene Wand} % (fold)
		Z.B.~in Wärmefluss in einem Haus bestehend aus:
		\begin{itemize}
			\item konvektivem Wärmeübergang von der Raumluft an die innere Oberfläche
			\item Wärmeleitung durch die Wand
			\item konvektivem Wärmeübergang von der Wand an die Aussenluft
		\end{itemize}

		\emphequation{equation*}{
			\frac{\partial^2 T}{\partial x^2} = 0 \qquad \text{mit } T(x=0) = T_1 \text{ und } T(x=L) = T_2
		}
		Durch zweimalige Integration und Einsetzen der Randbedingungen ergibt sich für den Temperaturverlauf:
		\emphequation{equation*}{
			T(x) = \frac{T_2 - T_1}{L} x + T_1
		}
		
		Wärmefluss durch Anwendung des Fourier'schen Gesetzes:
		\emphequation{equation*}{
			\dot Q_x'' = - \lambda \Diff{T}{x} = \lambda \frac{T_1 - T_2}{L} = \const \sunit{\watt\per\Square\metre}
		}
		Gesamter Wärmestrom durch die Fläche $A$:
		\emphequation{equation*}{
			\dot Q_x = \dot Q_x'' \cdot A = \frac{\lambda \cdot A}{L} (T_1 - T_2) \sunit{W}
		}
	% subsection: Wärmefluss durch die ebene Wand (end)
	\subsection{Der Begriff des Wärmeleitwiderstandes} % (fold)
		Zwischen Wärmeströmen und elektrischen Strömen existiert eine Analogie.
		
		Wärmeleitwiderstand:
		\emphequation{equation*}{
			R_\text{W} = \frac{1}{A} \cdot \frac{L}{\lambda} = \frac{1}{A} \cdot \frac{1}{\alpha}
		}
		\[
			I = \frac{1}{R} \cdot \Delta V \quad \Rightarrow \quad \dot Q = \frac{1}{R_\text{W}} \cdot \Delta T = \alpha \cdot A \cdot (T_\text{W} - T_\infty)
		\]
	% subsection: Der Begriff des Wärmeleitwiderstandes (end)
	\subsection{Ebene Wand mit Konvektion an den Oberflächen} % (fold)
		\paragraph{Die Biot-Zahl (BI)} % (fold)
			stellt das Verhältnis des Wär\-me\-leit\-wi\-der\-stan\-des im inneren eines Körpers (bez.~einer charakteristischen Ausdehnung $L$) zum äusseren konvektiven Wärme\-über\-gangswiderstand dar.
			\emphequation{equation*}{
				\mathrm{Bi} = \frac{\alpha\cdot L}{\lambda}
			}
			
			Die hauptsächliche Temperaturdifferenz findet man für…
			\begin{description}
				\item[grosse Bi-Zahl] innerhalb des Körpers
				\item[kleine Bi-Zahl] zwischen Oberfläche des Körpers und umgebendem Fluid
			\end{description}
		% paragraph: Die Biot-Zahl (BI) (end)
		
		Für den Temperaturverlauf ergibt sich:
		\emphequation{equation*}{
			T(x) = \frac{T_{\infty 2} - T_{\infty 1}}{2+\mathrm{Bi}} \cdot \parens{ \frac{\alpha}{\lambda} \cdot x + 1 } + T_{\infty 1}
		}
		
		Wärmefluss:
		\emphequation{equation*}{
			\dot Q_x'' = - \lambda \Diff{T}{x} = \frac{T_{\infty 1} - T_{\infty 2}}{2 + \mathrm{Bi}} \cdot \alpha = \const \sunit{\watt\per\Square\metre}
		}
		Gesamter Wärmestrom durch die Fläche $A$:
		\emphequation{equation*}{
			\dot Q_x = \dot Q_x'' \cdot A = \frac{T_{\infty 1} - T_{\infty 2}}{2 + \mathrm{Bi}} \cdot \alpha \cdot A = \const \sunit{W}
		}
		Oder durch Serieschaltung der Wider\-stände (zwei Ober\-flächen\-über\-gangs\-wider\-stände und einen Wärme\-leit\-wider\-stand der Wand):
		\[
			\dot Q_x = \frac{T_{\infty 1} - T_{\infty 2}}{\sum R} = \frac{T_{\infty 1} - T_{\infty 2}}{\frac{1}{\alpha \cdot A} + \frac{1}{\lambda \cdot A} + \frac{1}{\alpha \cdot A}} = \underbrace{\frac{1}{\frac{2}{\alpha} + \frac{L}{\lambda}}}_\text{$k$-Wert} \cdot A \cdot (T_{\infty 1} - T_{\infty 2})
		\]
		mit \[
			k = \frac{1}{\frac{2}{\alpha} + \frac{L}{\lambda}} \sunit{\watt\per\Square\metre\per\kelvin}
		\] als einen gesamten Wärmeübergangskoeffizienten (englisch: $U$-value).
		\emphequation{equation*}{
			\dot Q_x = k \cdot A \cdot \Delta T
		}
		
		Wärmeübergangskoeffizient $k$ für eine beliebige Wandkonstruktion:
		\emphequation{equation*}{
			k = \frac{1}{\displaystyle{\frac{1}{\alpha_\text{H}} + \sum_{i=1}^{n} \frac{L_i}{\lambda_i} + \frac{1}{\alpha_\text{K}}}} \sunit{\watt\per\Square\metre\per\kelvin}
		}
		Werte in Energienormen für Bauten:
		\begin{equation*}
			\alpha_\text{H} = \SI[fraction=nicefrac]{8}{\watt\per\Square\metre\per\kelvin} \ , \quad
			\alpha_\text{K} = \SI[fraction=nicefrac]{20}{\watt\per\Square\metre\per\kelvin}
		\end{equation*}
	% subsection: Ebene Wand mit Konvektion an den Oberflächen (end)
	\subsection{Wärmefluss durch eine Rohrisolation} % (fold)
		Innen $T_\text{$\infty$H}$,
		aussen $T_\text{$\infty$K}$.
		
		Temperaturverlauf in radialer Richtung:
		\emphequation{equation*}{
			T(r) = \frac{T_\text{$\infty$H} - T_\text{$\infty$K}}{\ln \nicefrac{r_1}{r_2}} \cdot \ln \frac{r}{r_2} + T_\text{$\infty$K}
		}
		
		Wärmestromdichte:
		\[
			\dot Q_r = \frac{1}{R} \cdot \Delta T
		\]
		mit dem Wärmeleitwiderstand
		\emphequation{equation*}{
			R = \frac{\ln \nicefrac{r_2}{r_1}}{2\pi\cdot\lambda\cdot L}
		}
		
		\paragraph{Rohrkonstruktion aus mehreren Schichten} % (fold)
			\emphequation{equation*}{
				\dot Q_r = k(r) \cdot A(r) \cdot ( T_\text{$\infty$H} - T_\text{$\infty$K} )
			}
			mit \[
				A(r) = 2\pi\cdot r \cdot L
			\] und
			\emphequation{equation*}{
				k(r) = \frac{1}{\displaystyle{
					\frac{r}{r_\text{H} \cdot \alpha_\text{H}} +
					r \cdot \sum_{i=1}^n \frac{\ln \nicefrac{r_{i+1}}{r_i}}{\lambda_i} +
					\frac{r}{r_\text{K} \cdot \alpha_\text{K}}
				}} \sunit{\watt\per\Square\metre\per\kelvin}
			}
		% paragraph: Rohrkonstruktion aus mehreren Schichten (end)
		
		\paragraph{Wärmeverlust pro Länge} % (fold)
			~
			
			Grundsätzlich ist man nicht an der lokalen Wärme\-fluss\-dichte interessiert, sondern am gesamten Wärmeverlust pro Länge.
			
			Man definiert:
			\emphequation{equation*}{
				\dot Q_r = k_\text{L} \cdot L \cdot (T_\text{$\infty$H} - T_\text{$\infty$K})
			}
			mit \textbf{$\boldsymbol{k_\text{L}}$ unabhängig von $\boldsymbol r$}:
			\emphequation{align*}{
				k_\text{L} &= k(r) \cdot 2\pi \cdot r = \frac{\dot Q_r}{L \cdot \Delta T} \\
				&= \frac{1}{\displaystyle{
					\frac{1}{r_\text{H} \cdot \alpha_\text{H}} +
					\sum_{i=1}^n \frac{\ln \nicefrac{r_{i+1}}{r_i}}{\lambda_i} +
					\frac{1}{r_\text{K} \cdot \alpha_\text{K}}
				}} \sunit{\watt\per\metre\per\kelvin}
			}
		% paragraph: Wärmeverlust pro Länge (end)
	% subsection: Wärmefluss durch eine Rohrisolation (end)
	\subsection{Wärmeleitung mit Quellen} % (fold)
		\subsubsection{Die ebene Wand mit Wärmequellen} % (fold)
			Temperaturverlauf ist \textbf{quadratisch}:
			\emphequation{equation*}{
				T(x) = - \frac{\dot Q_\text{Quellen}'''}{2\lambda} \cdot x^2 +
				\parens{\frac{T_2 - T_1}{L} + \frac{\dot Q_\text{Quellen}''' \cdot L}{2\lambda}} \cdot x + T_1
			}
			
			Lage des Extremums (Maxima falls Quelle positiv, Minima falls negativ):
			\emphequation{equation*}{
				x_e = \frac{L}{2} + \frac{T_2 - T_1}{L} \cdot \frac{\lambda}{\dot Q_\text{Quellen}'''}
			}
			
			Verteilung des Wärmeflusses:
			\emphequation{equation*}{
				\dot q'' = \dot Q_\text{Quellen}''' \cdot x - \lambda \cdot \frac{T_2 - T_1}{L} - \frac{\dot Q_\text{Quellen}''' \cdot L}{2}
			}
			
			Wärmeflüsse an den Oberflächen:
			\emphequation{align*}{
				\dot q_{x=0}'' &= - \lambda \cdot \frac{T_2 - T_1}{L} - \frac{\dot Q_\text{Quellen}''' \cdot L}{2} \\
				\dot q_{x=L}'' &= \frac{\dot Q_\text{Quellen}''' \cdot L}{2} - \lambda \cdot \frac{T_2 - T_1}{L}
			}
			
			Die Summe der austretenden Wärmeflüsse muss nach der Energieerhaltung gleich der Gesamtheit der eingeschlossenen Wärmequellen sein:
			\[
				\dot q_{x=L}'' - 
				\dot q_{x=0}'' = \dot Q_\text{Quellen}''' \cdot L
			\]
		% subsubsection: Die ebene Wand mit Wärmequellen (end)
		\subsubsection{Zylinder mit Wärmequellen} % (fold)
			Temperaturverlauf:
			\emphequation{equation*}{
				T(r) = \frac{\dot Q_\text{Quellen}''' \cdot r_0^2}{4\lambda} \cdot \parens{1-\frac{r^2}{r_0^2}} + T_0
			}
			\begin{itemize}
				\item[$r_0$:] Aussenradius
			\end{itemize}

			Temperaturmaxima:
			\emphequation{equation*}{
				T_\text{max} = T(r=0) = T_0 + \frac{\dot Q_\text{Quellen}''' \cdot r_0^2}{4\lambda}
			}
			
			Wärmefluss:
			\emphequation{equation*}{
				\dot q_0'' = -\lambda \cdot \left.\parens{\Diff T r}\right|_{r=r_0} = \frac{\dot Q_\text{Quellen}''' \cdot r_0}{2} \sunit{\watt\per\Square\metre}
			}
		% subsubsection: Zylinder mit Wärmequellen (end)
	% subsection: Wärmeleitung mit Quellen (end)
	\subsection{Wärmeleitung in Rippen} % (fold)
		Rippen dienen zur Vergrösserung der Wärmetauscherfläche.
		
		Ist die Dicke im Verhältnis zur Länge der Rippen klein, können diese mit eindimensionaler Wärmeleitung behandelt werden.
		
		\subsubsection{Allgemeine Energieerhaltungsgleichung für Rippen} % (fold)
			\begin{itemize}
				\item Bewegung in Längsrichtung mit Geschwindigkeit $u$
				\item In Querrichtung durch Fluid angeströmt (konvektiver Wärmeübergang mit $\alpha$)
				\item Querschnittsfläche $S(x)$, welche von $x$ abhängen kann
				\item Umfang $P$
				\item Temperaturverläufe innerhalb der der Querschnittsfläche können vernachlässigt werden
			\end{itemize}
			
			\emphequation{align*}{
				\rho c \Part{}{t}\parens{\Diff V x T} &= \lambda \Part{}{x} \parens{S \Part T x} - \rho c \Part{}{x} \parens{SuT}\\
				& \quad \, - \Diff A x \alpha \parens{ T - T_\infty} + \dot Q_\text{Quellen}''' \Diff V x \\
				& \, \quad - \dot Q_\text{Strahlung}'' \Diff A x
			}
			mit \[
				\dot Q_\text{Strahlung}'' = \sigma \epsilon (T^4 - T_\infty^4)
			\]
		% subsubsection: Allgemeine Energieerhaltungsgleichung für Rippen (end)
		\subsubsection{Rippen an Wärmetauschern} % (fold)
			\begin{itemize}
				\item Querschnittsfläche $= \const$
				\item Rippe bewegt sich nicht $(u = 0)$
				\item stationär $(\partial/\partial t = 0)$
				\item keine Quellen
				\item keine Strahlung
			\end{itemize}
			
			Die allgemeine Energieerhaltungsgleichung vereinfacht sich zu:
			\emphequation{equation*}{
				\frac{\diff^2 \theta}{\diff x^2} - m^2 \cdot \theta = 0
			}
			mit \textbf{Rippenparameter}
			\[
				m^2 = \frac{\alpha \cdot P}{\lambda \cdot S}
			\]
			und \textbf{Übertemperatur}
			\[
				\theta = T - T_\infty
			\]
			
			Zweimaliges Integrieren ergibt:
			\emphequation{equation*}{
				\theta(x) = C_1 \cdot \eu^{mx} + C_2 \cdot \eu^{-mx}
			}
			
			\paragraph{Randbedingungen} % (fold)
				~
				
				Zwei Randbedingungen nötig:
				\begin{description}
					\item[1. Randbedingung:] Temperatur des Rippenfusses bekannt:
						\begin{align*}
							T(0) = T_\text{F} \quad \Rightarrow \quad \theta(0) = T_\text{F} - T_\infty = \theta_\text{F} = C_1 + C_2
						\end{align*}
					\item[2. Randbedingung:] mehrere Möglichkeiten:
						\begin{description}
							\item[Rippe bei $\boldsymbol{x = L}$ isoliert:] (gute Näherung zur Realität und einfacher als nächster Punkt)
								\emphequation{equation*}{
									\theta(x) = \theta_\text{F} \cdot \frac{\cosh(m \cdot (L-x))}{\cosh (m\cdot L)}
								}
							\item[Konvektiver Wärmeübergang am Rippenende:]
								\[
									C_1 \cdot m \cdot \eu^{mL} - C_2 \cdot m \cdot \eu^{-mL} = - \frac{\alpha}{\lambda} \cdot \parens{C_1 \cdot \eu^{mL} + C_2 \cdot \eu^{-mL}}
								\]
							\item[Temperatur des Rippenendes bekannt:]
								\begin{align*}
									T(L) &= T_\text{K} \\
									\Rightarrow \quad \theta(L) &= T_\text{K} - T_\infty \\
									&= \theta_\text{K} = C_1 \cdot \eu^{mL} + C_2 \cdot \eu^{-mL}
								\end{align*}
							\item[Rippe ist sehr lang (kein Wärmefluss bis zum Kopf):]
								\[
									C_1 = 0
								\]
						\end{description}
				\end{description}
			% paragraph: Randbedingungen (end)
			
			\textbf{Wärmeübertragungsleistung} (Wärmefluss, der am Fuss in die Rippe hineinfliesst):
			\emphequation{equation*}{
				\dot Q_\text{F} = \lambda \cdot S \cdot \theta_\text{F} \cdot m \cdot \tanh(m\cdot L)
			}
			
			Eine Rippe erreicht ihre \textbf{maximale Übertragungsleistung} bei $m\cdot L = 2$. Eine weitere Verlängerung erhöht die Wärme\-über\-tragung kaum.
			
			Ein \textbf{effizienter Materialeinsatz} ist für $m\cdot L = 1$ gewähr\-leistet.
			\[
				L_\text{optimal} < \frac{1}{m} = \sqrt \frac{\lambda \cdot S}{\alpha \cdot P}
			\]
			Je grösser $\lambda$ (um so kleiner $m\cdot L$) desto höher die Temperatur am Rippenende.
		% subsubsection: Rippen an Wärmetauschern (end)
		\subsubsection{Der Rippenwirkungsgrad} % (fold)
			\emphequation{align*}{
				\eta_\text{R} &= \frac{\text{übertragene Wärmemenge}}{\text{maximal übertragbare Wärmemenge}} \\
				&= \frac{\dot Q_\text{F}}{\dot Q_\text{M}} = \frac{\lambda \cdot S \cdot \theta_\text{F} \cdot m \cdot \tanh(m\cdot L)}{\alpha \cdot P \cdot L \cdot \theta_\text{F}} \\
				&= \frac{\tanh(m\cdot L)}{m\cdot L}
			}
			\begin{align*}
				\eta_\text{R}(m\cdot L = 1) &< 80\% \\
				\eta_\text{R}(m\cdot L = 2) &< 50\%
			\end{align*}
		% subsubsection: Der Rippenwirkungsgrad (end)
		\subsubsection{Kreisförmige Rippen} % (fold)
			Für den Fall des isolierten Rippenkopfes:
			\emphequation{equation*}{
				\theta(r) = \theta_\text{F} \frac{I_0(m \cdot r) \cdot K_1(m\cdot r_2) + I_1(m\cdot r_2) \cdot K_0(m\cdot r)}{I_0(m \cdot r_1) \cdot K_1(m\cdot r_2) + I_1(m\cdot r_2) \cdot K_0(m\cdot r_1)}
			}
			mit \[
				m^2 = \frac{2\alpha}{\lambda \cdot d}
			\]
			wobei $I_{0,1}$ und $K_{0,1}$ Besselfunktionen 0.~und 1.~Ordnung sind.
		% subsubsection: Kreisförmige Rippen (end)
	% subsection: Wärmeleitung in Rippen (end)
% section: Stationäre eindimensionale Wärmeleitung (end)