%!TEX root = ../Thermodynamik II (LTNT).tex

\section{Grundlagen der Konvektion} % (fold)
		Mittlerer $\alpha$-Wert:
		\emphequation{equation*}{
			\overline\alpha = \frac{1}{A_0} \int_{A_0} \alpha \diff A_0
		}
		Für den Fall einer ebenen Platte:
		\emphequation{equation*}{
			\overline\alpha = \frac{1}{L} \int_{0}^L \alpha(x) \diff x
		}

	\subsection{Erhaltungsgleichung für 2D-Konvektion} % (fold)
		\subsubsection{Die Kontinuitätsgleichung} % (fold)
			Im stationären Fall für ein inkompressibles Fluid ($\rho = \const$):
			\emphequation{equation*}{
				\frac{\partial u}{\partial x} + \frac{\partial v}{\partial y} = 0
			}
			Allgemein in Vektorform:
			\[
				\Div(\rho \cdot \vec v) = 0
			\]
		% subsubsection die_kontinuitätsgleichung (end)
		\subsubsection{Impulserhaltungsgleichung} % (fold)
			 Die Änderung des \textbf{Impulsstromes} im Kontrollvolumen ist gleich den wirkenden \textbf{Oberflächenkräften} und der \textbf{Volumenkraft}.

			Durch Umformung der \emph{Navier-Stokes-Gleichung} erhält man für konstante Stoffeigenschaften:
			\emphequation{align*}{
				\rho \parens{
					u \frac{\partial u}{\partial x} + v \frac{\partial u}{\partial y}
				} &= - \frac{\partial P}{\partial x} + \mu \parens{
					\frac{\partial^2 u}{\partial x^2} + v \frac{\partial^2 u}{\partial y^2}
				} + \rho g_x \\
				\rho \parens{
					u \frac{\partial v}{\partial x} + v \frac{\partial v}{\partial y}
				} &= - \frac{\partial P}{\partial y} + \mu \parens{
					\frac{\partial^2 v}{\partial x^2} + v \frac{\partial^2 v}{\partial y^2}
				} + \rho g_y
			}

			In Vektorform:
			\[
				\rho \parens{\vec V \cdot \nabla} \vec V = - \vec V P + \mu \nabla^2 \cdot \vec V + \rho \vec g
			\]
			wobei:
			$
				\vec V = \vec i \cdot u + \vec j \cdot v
			$
		% subsubsection impulserhaltungsgleichung (end)
		\subsubsection{Die Energieerhaltungsgleichung} % (fold)
			Vollständige Form der Energiegleichung:
			\emphequation{equation*}{
				\rho c \parens{
					u \frac{\partial T}{\partial x} + v \frac{\partial T}{\partial y}
				} = \frac{\partial}{\partial x} \parens{\lambda \frac{\partial T}{\partial x}} +
				\frac{\partial}{\partial y} \parens{\lambda \frac{\partial T}{\partial y}} +
				\mu\Phi + \dot Q_\text{Q}'''
			}
			wobei: \[
				\Phi = \parens{
					\frac{\partial u}{\partial y} + \frac{\partial v}{\partial x}
				}^2 - 4 \parens{
					\frac{\partial u}{\partial x} \cdot \frac{\partial v}{\partial y}
				}
			\]

			Falls $\lambda = \const$ und $\dot Q_Q''' = 0$:
			\emphequation{equation*}{
				\parens{
					u \cdot \frac{\partial T}{\partial x} + v \cdot \frac{\partial T}{\partial y}
				} = a \parens{
					\frac{\partial^2 T}{\partial x^2} + \frac{\partial^2 T}{\partial y^2}
				}
			}
			In Vektorform:
			\[
				\vec V \cdot \grad T = a\cdot \nabla^2T
			\]

			\paragraph{Anmerkung:} % (fold)
				Der Term $\Phi$ beinhaltet nicht lineare Terme, die das chaotische Verhalten der Turbulenz beschreiben. Dessen Weglassung bedeutet, dass wir uns auf laminare Strömungen beschränken.
			% paragraph anmerkung_ (end)
		% subsubsection die_energieerhaltungsgleichung (end)
	% subsection erhaltungsgleichung_für_2d_konvektion (end)
% section grundlagen_der_konvektion (end)
