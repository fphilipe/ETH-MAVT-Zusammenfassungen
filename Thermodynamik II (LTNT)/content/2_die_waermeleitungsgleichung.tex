%!TEX root = ../Thermodynamik II (LTNT).tex

\section{Die Wärmeleitungsgleichung} % (fold)
	\subsection{Die Energieerhaltungsgleichung} % (fold)
		\emphequation{equation*}{
			\rho c \frac{\partial T}{\partial t} = \nabla \cdot (\lambda \cdot \nabla T) + \dot Q_\text{Quellen}'''
		}

		Falls $\boldsymbol{\lambda = \const}$:
		\emphequation{equation*}{
			\frac 1 a \frac{\partial{T}}{\partial{t}} = \nabla^2 T + \frac 1 \lambda \cdot \dot Q_\text{Quellen}''' \qquad \text{mit } a = \frac{\lambda}{\rho\cdot c} \unit{\Square\metre\per\second}
		}

		Falls \textbf{stationär} und $\dot{Q}'''_\text{Quellen} = 0$:
		\emphequation{equation*}{
			\nabla^2 T = 0
		}

		\paragraph{Zylinder-Koordinaten} % (fold)
			\[
				\nabla = \begin{bmatrix}
					\frac{\partial}{\partial r},\, \frac{1}{r} \frac{\partial}{\partial \theta},\, \frac{\partial}{\partial z}
				\end{bmatrix}
				\ , \quad
				\nabla^2 = \begin{bmatrix}
					\frac{1}{r}\frac{\partial}{\partial r}\parens{r \frac{\partial}{\partial r}}+\frac{1}{r^2} \frac{\partial^2}{\partial \theta^2} + \frac{\partial^2}{\partial z^2}
				\end{bmatrix}
			\]
			\emphequation{equation*}{
				\rho c \Part{T}{t} = \frac{1}{r}\Part{}{r}\parens{\lambda r \Part{T}{r}} + \frac{1}{r^2} \Part{}{\theta}\parens{\lambda \Part{T}{\theta}} + \Part{}{z} \parens{\lambda \Part{T}{z}} + \dot Q_\text{Q}'''
			}

			Falls $\boldsymbol{\lambda = \const}$:
			\emphequation{equation*}{
				\frac{1}{a}\Part{T}{t} = \frac{1}{r}\Part{}{r}\parens{r \Part{T}{r}} + \frac{1}{r^2} \frac{\partial^2 T}{\partial \theta^2} + \frac{\partial^2 T}{\partial z^2} + \frac{\dot Q_\text{Q}'''}{\lambda}
			}
		% paragraph: Zylinder-Koordinaten (end)

		\paragraph{Kugel-Koordinaten} % (fold)
			\begin{align*}
				\nabla &= \begin{bmatrix}
					\frac{\partial}{\partial r},\, \frac{1}{r} \frac{\partial}{\partial \phi},\, \frac{1}{r\sin{\phi}}\frac{\partial}{\partial \theta}
				\end{bmatrix}
				\ , \\
				\nabla^2 &= \begin{bmatrix}
					\frac{1}{r^2}\frac{\partial}{\partial r}\parens{r^2 \frac{\partial}{\partial r}}+\frac{1}{r^2\sin\phi} \frac{\partial}{\partial \phi}\parens{\sin\phi \frac{\partial}{\partial \phi}} + \frac{1}{r^2 \sin^2\phi} \frac{\partial^2}{\partial \theta^2}
				\end{bmatrix}
			\end{align*}
			\emphequation{equation*}{
				\begin{split}
					\rho c \Part{T}{t} &= \frac{1}{r^2}\Part{}{r}\parens{\lambda r^2 \Part{T}{r}} + \frac{1}{r^2\sin^2\phi} \Part{}{\theta}\parens{\lambda \Part{T}{\theta}} \\
					&\quad \ + \frac{1}{r^2\sin\phi}\Part{}{\phi} \parens{\lambda \sin\phi \Part{T}{\phi}} + \dot Q_\text{Q}'''
				\end{split}
			}

			Falls $\boldsymbol{\lambda = \const}$:
			\emphequation{equation*}{
				\begin{split}
					\frac{1}{a}\Part{T}{t} &= \frac{1}{r^2}\Part{}{r}\parens{r^2 \Part{T}{r}} + \frac{1}{r^2\sin^2\phi} \frac{\partial^2 T}{\partial \theta^2} \\
					& \quad \ + \frac{1}{r^2\sin\phi}\Part{}{\phi}\parens{\sin\phi\Part{T}{\phi}} + \frac{\dot Q_\text{Q}'''}{\lambda}
				\end{split}
			}
		% paragraph: Kugel-Koordinaten (end)
	% subsection: Die Energieerhaltungsgleichung (end)

	\subsection{Rand- und Anfangsbedingungen} % (fold)
		\begin{itemize}
			\item Temperatur der Oberfläche ist gegeben:
				\[
					T(x=0,t) = T_0(t)
				\]
			\item Wärmestrom an der Oberfläche ist bekannt:
				\[
					-\lambda \left.\parens{\Part T x}\right|_{x=0} = \dot q''(0)
				\]
				Bei ideal isolierten Körpern verschwindet der Wärmestrom auf der Oberfläche:
				\[
					\dot q_0'' = 0 \quad \text{bzw.} \quad \left.\parens{\Part T x}\right|_{x=0} = 0
				\]
			\item Konvektiver Wärmeübergang auf der Aussenfläche vorgegeben:
				\[
					\underbrace{\lambda \left.\parens{\Part T x}\right|_{x=0}}_\text{Körper} = \underbrace{\alpha [T(0,t) - T_\infty] \vphantom{\left.\parens{\Part T x}\right|_{x=0}}}_\text{umgebendes Fluid}
				\]
			\item Anfangstemperatur gegeben:
			\[
				T(x,t=0) = T_0
			\]
		\end{itemize}
	% subsection: Rand- und Anfangsbedingungen (end)
% section: Die Wärmeleitungsgleichung (end)
