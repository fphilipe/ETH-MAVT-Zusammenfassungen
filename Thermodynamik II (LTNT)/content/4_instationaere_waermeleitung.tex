%!TEX root = ../Thermodynamik II (LTNT).tex

\section{Instationäre Wärmeleitung} % (fold)
	\subsection{Instationäre, vom Raum un\-ab\-hängige Wärme\-leitung} % (fold)
		Voraussetzung, damit die innere Temperaturverteilung eines Körpers vernachlässigt werden kann:
		\emphequation{equation*}{
			\frac{T_\text{i} - T_0}{T_0 - T_\infty} = \mathrm{Bi} \ll 1
		}
		Erfüllt für:
		\emphequation{equation*}{
			\alpha \cdot L \ll 1 \text{ (kleine Körper)} \quad \text{oder} \quad \lambda \gg 1 \text{ (metallische Körper)}
		}
		$\Rightarrow$ \textbf{Temperatur hängt nur von der Zeit ab}

		\paragraph{Abkühlung eines heissen Körpers} % (fold)
			\emphequation{equation*}{
				\frac{T - T_\infty}{T_\text{i} - T_\infty} = \exp\parens{- \frac{\alpha \cdot A}{M \cdot c} \cdot t}
				 = \exp\parens{- \frac t \tau}
			}
			mit Zeitkonstante $\tau$, in der der ursprüngliche Wert auf $1/\eu$ abfällt:
			\[
				\frac{1}{\tau} = \frac{\alpha \: A}{M\: c} \sunit{\per\second}
			\]
			wobei $M\:c$ ein ``lumped parameter'' sein kann:
			\[
				(M\:c) = \sum_i (m\:c)_i
			\]

			Nach dem vollständigen Ausgleich beträgt die Übertragene Wärmemenge:
			\emphequation{equation*}{
				Q_\text{tot} = M \: c \cdot (T_\text{i} - T_\infty)
			}
		% paragraph: Abkühlung eines heissen Körpers (end)
	% subsection: Instationäre, vom Raum unabhängige Wärmeleitung (end)

	\subsection{Instationäre Wärmeleitung in der halbunendlichen Wand} % (fold)
		Erhaltungsgleichung für den eindimensionalen zeit\-ab\-hängigen Fall:
		\[
			\frac{1}{a} \Part T t = \frac{\partial^2 T}{\partial x^2}
		\]

		\paragraph{Plötzlicher Temperatursprung an der Oberfläche} % (fold)
			Anfangs- und Randbedingungen:
			\begin{align*}
				T_\text{i} &= T(t<0,x \ge 0) \\
				&= T(t=0,x > 0) \\
				&= T(t \to \infty, x \to \infty) \\
				T_\text{S} &= T(t = 0 , x = 0)
			\end{align*}
			Lösung:
			\emphequation{equation*}{
				\frac{T - T_\text{S}}{T_\text{i} - T_\text{S}} = \frac{2}{\sqrt \pi} \int_0^\eta \eu^{-u^2} \diff u \equiv \erf(\eta) \equiv 1-\erfc(\eta)
			}
			mit
			\emphequation{equation*}{
				\eta = \frac{x}{\sqrt{4\cdot a \cdot t}}
			}
			Diffusionslänge:
			\emphequation{equation*}{
				x_\text{D}(t) = \sqrt{a \cdot t}
			}
			Wärmestrom:
			\emphequation{equation*}{
				\dot q_\text{S}'' = \frac{\lambda \cdot (T_\text{S} - T_\text{i})}{\sqrt{\pi\cdot a \cdot t}} \sim \frac{1}{\sqrt t}
			}
		% paragraph: Plötzlicher Temperatursprung an der Oberfläche (end)

		\paragraph{Konstanter Wärmestrom auf der Oberfläche} % (fold)
			\[
				\dot q_\text{S}'' (t) = \dot q_0'' = \const
			\]
			\emphequation{equation*}{
				T(x,t) - T_\text{i} = \frac{\dot q_0''}{\lambda}  \left[
					\sqrt \frac{4  a  t}{\pi} \cdot \exp \parens{\frac{-x^2}{4 a  t}}
					- x \cdot \erfc\parens{\frac{x}{2\sqrt{a t}}}
				\right]
			}
		% paragraph: Konstanter Wärmestrom auf der Oberfläche (end)

		\paragraph{Konvektiver Wärmeübergang auf der Oberfläche} % (fold)
			\emphequation{gather*}{
				\frac{T(x,t) - T_\text{i}}{T_\infty - T_\text{i}} = \erfc\parens{\frac{x}{2\sqrt{a t}}} \\
				- \exp\parens{\frac{\alpha  x}{\lambda} + \frac{\alpha^2  a  t}{\lambda^2}} \cdot \erfc\parens{\frac{x}{2 \sqrt{a t}} + \frac{\alpha  \sqrt{a  t}}{\lambda}}
			}
		% paragraph: Konvektiver Wärmeübergang auf der Oberfläche (end)
	% subsection: Instationäre Wärmeleitung in der halbunendlichen Wand (end)
% section: Instationäre Wärmeleitung (end)
