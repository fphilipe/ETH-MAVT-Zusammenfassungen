%!TEX root = ../Thermodynamik II (LTNT).tex

\section{Mechanismen der Wärmeübertragung} % (fold)
	\subsection{Die Arten der Wärmeübertragung} % (fold)
		\begin{description}
			\item[Leitung (Conduction):] in jeglicher Art von Materie, hauptsächlich in Festkörpern
			\item[Konvektion (Convection):] Wärmetransport, der mit fliessender Materie verbunden ist (Gase, Flüssigkeiten)
			\item[Strahlung (Radiation):] Wärmetransport in Form von elektromagnetischer Strahlung (nicht an Materie gebunden)
		\end{description}
	% subsection: Die Arten der Wärmeübertragung (end)

	\subsection{Wärmeleitung} % (fold)
		Kann man auch als \emph{Diffusion von thermischer Energie} auffassen.

		\paragraph{Das Gesetz von Fourier (Fourier's Law)} % (fold)
			\emphequation{equation*}{
				\dot q'' = - \lambda \Diff T x
			}
			Wärmestromdichte ist proportional zum Temperaturgradienten.
			\begin{itemize}
				\item[$\dot q''$:] Wärmestromdichte (Wärmestr.~durch Fläche) \niceunit{\watt\per\metre\squared}
				\item[$\lambda$:] Wärmeleitfähigkeit (Materialabhängig) \niceunit{\watt\per\metre\per\kelvin}
					\begin{center}
						\begin{tabular}{l|r@{}l}
							\textbf{Material} & \multicolumn{2}{l}{\textbf{Wärmeleitfähigkeit} \niceunit{\watt\per\metre\per\kelvin}} \\
							\midrule
							Kupfer & 380 & \\
							Glas & 1&.05 \\
							Wasser & 0&.599 \\
							Luft & 0&.024 \\
							Styropor & 0&$.03 - 0.05$
						\end{tabular}
					\end{center}
			\end{itemize}
		% paragraph: Das Gesetz von Fourier (Fourier's Law) (end)

		\subsubsection{Konvektion} % (fold)
			\begin{equation*}
				\dot q'' = \frac{\dot{Q}}{A} = \alpha (T_\text{H} - T_\text{K})
			\end{equation*}

			$\alpha$ ist der konvektive Wärmeübertragungskoeffizient (englisch $h$) und hängt von verschiedenen Faktoren ab:
			\begin{tightitemize}
				\item Art der Strömung (laminar, turbulent)
				\item Oberflächenrauhigkeit
				\item Geometrie
				\item Art des Fluids
				\item Phasenübergänge
			\end{tightitemize}

			Je nach Randbedingungen unterscheidet man \emph{vier Arten} des konvektiven Wärmeübergangs:

			\begin{tightitemize}
				\item Erzwungene Konvektion
				\item Natürliche Konvektion
				\item Sieden
				\item Kondensation
			\end{tightitemize}

			\begin{tabular}{lr|r@{$ - $}l}
				\multicolumn{2}{l}{\textbf{Art der Konvektion}} & \multicolumn{2}{l}{$\boldmath{\alpha}$ \niceunit{\watt\per\metre\squared\per\kelvin}} \\
				\midrule
				\multicolumn{2}{l}{\textbf{Natürliche Konvektion}} & \multicolumn{2}{l}{} \\
				& Gase & 2 & 25 \\
				& Flüssigkeiten & 50 & $1000$ \\
				\midrule
				\multicolumn{2}{l}{\textbf{Erzwungene Konvektion}} & \multicolumn{2}{l}{} \\
				& Gase & 25 & 250 \\
				& Flüssigkeiten & 50 & 20000 \\
				\midrule
				& \textbf{Siedendes Wasser} & 2000 & 25000 \\
				\midrule
				& \textbf{Kondensation von Wasserdampf} & 5000 & 100000 \\
			\end{tabular}
		% subsubsection: Konvektion (end)

		\subsubsection{Wärmestrahlung} % (fold)
			Ein Körper, der sämtliche auf ihn treffende Strahlung absorbiert, nennst man einen idealen \textbf{schwarzen Körper}.
			Dieser zeigt auch das maximal mögliche Emissionsvermögen $E_b$, welches durch das \textbf{Stefan-Boltzmann Gesetz} beschreiben wird:
			\begin{equation*}
				E_b = \sigma \cdot T^4 \qquad \text{mit } \sigma = 5.67\cdot 10^{-8} \si[fraction=frac]{\watt\per\metre\squared\per\kelvin\tothe{4}}
			\end{equation*}
			Für nicht ideale Körper (graue Körper) reduziert es sich um einen Faktor $\epsilon$:
			\begin{equation*}
				E = \epsilon\cdot\sigma\cdot T^4 \qquad \text{mit } 0 < \epsilon < 1
			\end{equation*}
		% subsubsection: Wärmestrahlung (end)
	% subsection: Wärmeleitung (end)
% section: Mechanismen der Wärmeübertragung (end)
