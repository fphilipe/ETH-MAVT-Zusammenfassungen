%!TEX root = ../Thermodynamik II (LTNT).tex

\section{Erzwungene Konvektion an durchströmten Kanälen} % (fold)
	\subsection{Hydrodynamische Betrachtung} % (fold)
		Geschwindigkeitsprofil der Strömung nach der Einlauflänge:
		\emphequation{equation*}{
			u(r) = (r_0^2 - r^2) \frac{p_1 - p_2}{4\mu l} = - \Diff p x \cdot \frac{r_0^2 - r^2}{4 \mu}
		}

		Durchschnittsgeschwindigkeit (konstant über ganze Länge):
		\emphequation{equation*}{
			u_m = \frac{1}{A} \int_0^L \int_0^{2\pi} u(r) \cdot r \diff \theta \diff x
		}

		\[
			\dot{m} = \rho u_m A
			\ ,\quad
			\mathrm{Re}_D = \frac{\rho u_m D}{\mu}
			\ ,\quad
			\text{mit } \mathrm{Re}_{D,krit} \approx 2300
		\]

		Bestimmung der Einlauflänge $X_{d,h}$:
		\[
			\parens{\frac{X_{d,h}}{D}}_\text{lam} \approx 0.05 \cdot \mathrm{Re}_D
			\ ,\quad
			10 \leq \parens{\frac{X_{d,h}}{D}}_\text{turb} \leq 60
		\]

		Druckverlust und Widerstandsbeiwert (Moody friction factor) für laminare Strömungen:
		\[
			-\Diff p x = \frac \lambda D \cdot \frac \rho 2 \cdot u_m^2
			\ ,\quad
			\lambda = \frac{64}{\mathrm{Re}_D}
		\]
	% (end)

	\subsection{Thermische Betrachtung} % (fold)
		Eintrittslänge des Temperaturprofils:
		\[
			\parens{\frac{X_{d,T}}{D}}_\text{lam} \approx 0.05 \cdot \mathrm{Re}_D \cdot \mathrm{Pr}
		\]

		Mittlere Temperatur:
		\[
			T_m = \frac{2}{u_m \cdot r_0^2} \int_0^{r_0} u \cdot T(r) \cdot r \diff r
		\]

		Wärmestrom zwischen Wand und Fluid:
		\[
			\dot{q}'' = \alpha (T_s - T_m)
			\ ,\ \text{wobei } \alpha = \const\,, \ x \geq X_{d,T}
		\]

		Änderung der Temperatur entlang des Rohres:
		\[
			\Diff{T_m}{x} = \frac{U}{c_p \dot m} \cdot \dot{q}''
			\ ,\quad \text{mit } U = \pi D
		\]

		Es existieren zwei Fälle:
		\begin{description}
			\item[$\dot{q}'' = \const$:] \[
				T_m(x) = T_m(0) + \frac{ \dot{q}'' \cdot U }{ c_p \dot m } \cdot x
			\]
			\item[$T_s = \const$:] \begin{gather*}
				\frac{\Delta T(x)}{\Delta T_e} = \frac{ T_s - T_m(x) }{ T_s - T_{m}(L) } = \exp \parens{ - \frac{ U \overline{\alpha}_x }{ c_p \dot m  } \cdot x } \\
				\overline{\alpha}_x = \frac 1 x \int_0^x \alpha \diff x
			\end{gather*}
			\[
				\overline{\Delta T} = \frac 1 L \int_0^L \Delta T(x) \diff x = \frac{ \Delta T_a - \Delta T_e }{ \ln \nicefrac{ \Delta T_a }{ \Delta T_e } }
			\]
			Falls $X_{d,T}$ kurz ist gegenüber $x$, kann man deren Einfluss vernachlässigen:
			\[ \overline{\alpha}_x = \overline{\alpha}_L = \overline{\alpha} \]
			\item[$T_\infty = \const$:] Gleiche wie bei $T_s = \const$. Es muss lediglich $T_s$ durch $T_\infty$ und $\overline{\alpha}$ durch $\overline{k}_L$ ersetzt werden.
			\[
				\dot Q_s = \overline{k}_L \cdot L \cdot \overline{\Delta T}
			\]
		\end{description}
	% subsection Thermische Betrachtung (end)

	\subsection{Wärmeübergangskoeffizient bei laminarer Strömung} % (fold)
		Für $\dot{q}'' = \const$ gilt:
		\emphequation{equation*}{
			T(r) = T_s - \frac{ 2 u_m r_0^2 }{a} \cdot \Diff{T_m}{x} \cdot \parens{ \frac{3}{16} + \frac{1}{16} \parens{\frac{r}{r_0}}^4 - \frac{1}{4} \parens{\frac{r}{r_0}}^2}
		}
		wobei $a = \frac{\lambda}{\rho \cdot c}$ die Temperaturleitfähigkeit ist.

		\paragraph{Nusselt-Zahl} % (fold)
			\begin{description}
				\item[Bei konstantem Wärmestrom:] \[
					\mathrm{Nu}_D = \frac{ \alpha \cdot D }{ \lambda } = 4.36
				\]
				\item[Bei konstanter Oberflächentemperatur:] \[
					\mathrm{Nu}_D = \frac{ \alpha \cdot D }{ \lambda } = 3.66
				\]
				\item[Im Einflaufbereich:] \[
					\overline{\mathrm{Nu}}_D = \frac{ \alpha \cdot D }{ \lambda } = \left[
						3.66 + \frac{
							0.067 \cdot \parens{ \mathrm{Re}_D \cdot \mathrm{Pr} \cdot \frac d L }^{1.33}
						}{
							1 + 0.1 \cdot \mathrm{Pr} \cdot \parens{ \mathrm{Re}_D \cdot \frac d L }^{0.83}
						} \cdot \parens{\frac{\mu}{\mu_w}}^{0.14}
					\right]
				\]
			\end{description}
		% paragraph Nusselt-Zahl (end)
	% subsection Wärmeübergangskoeffizient bei laminarer Strömung (end)

	\subsection{Wärmeübergangskoeffizient bei turbulenter Strömung} % (fold)
		Folgender experimenteller Ansatz für die Nusselt-Zahl gilt sowohl für konstante Oberflächentemperatur als auch für konstanter Wärmestrom. Der Einlauf wir durch den letzten Term berücksichtigt:
		\begin{align*}
			\overline{\mathrm{Nu}}_D =&\ 0.0235 \cdot ( \mathrm{Re}_D^{0.8} - 230 ) \cdot ( 1.8 \cdot \mathrm{Pr}^0.3 - 0.8 ) \\
			& \cdot \parens{\frac{\mu}{\mu_w}}^{0.14} \cdot \left[
				1 + \parens{\frac d L}^{\nicefrac 2 3}
			\right]
		\end{align*}

		Falls $\mathrm{Re_D} \in (3000,10^5)$, $\mathrm{Pr} \in (0.6, 500)$ und $L/d > 40$, dann gilt die einfachere Beziehung:
		\[
			\overline{\mathrm{Nu}}_D = 0.027 \cdot \mathrm{Re}_D^{0.8} \cdot \mathrm{Pe}^{1/3} \cdot \parens{\frac{\mu}{\mu_w}}^{0.14}
		\]
	% subsection Wärmeübergangskoeffizient bei turbulenter Strömung (end)
% (end)
