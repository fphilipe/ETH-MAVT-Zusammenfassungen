%!TEX root = ../Thermodynamik I.tex

\section{Der Erste Hauptsatz} % (fold)
	\subsection{Energieformen} % (fold)
		\begin{tabular}{@{}ll@{}}
			$KE$ & kinetische Energie\\
			$PE$ & potentielle Energie\\
			$U$  & innere Energie (Wärme, chem.~Bindungsenergie)\\
			$u$  & spezifische innere Energie $\left(u=\nicefrac{U}{m}\right)$\\
			$E$  & gesamt Energie\\
			$Q$  & Wärme\\
			$W$  & Arbeitsleistung
		\end{tabular}
		
		\[
			\begin{array}{r@{:\:}l@{\quad}r@{:\:}l@{\quad\text{\dots}}l}
				Q<0 & \text{\textbf{vom} Sys.} & Q>0 & \text{\textbf{zum} Sys.} & \text{transferiert}\\
				W<0 & \text{\textbf{am} Sys.} & W>0 & \text{\textbf{vom} Sys.} & \text{verrichtet}\\
			\end{array}
		\]
		
		\begin{align*}
			E &= KE + PE + U
		\end{align*}
		\begin{empheq}[box=\shadowbox*]{equation*}
			\Delta E = \Delta KE + \Delta PE + \Delta U
		\end{empheq}
		\begin{align*}
			E_2 &= E_1 + \underbrace{Q_{12}}_{\mathclap{\substack{
				\text{zugeführte} \\ \text{Wärme}
			}}} - \overbrace{W_{12}}^{\mathclap{\substack{
				\text{abgeführte} \\ \text{Arbeitsleistung}
			}}}
		\end{align*}
	% subsection: Energieformen (end)
	\subsection{1. Hauptsatz} % (fold)
		\begin{empheq}[box=\shadowbox*]{align*}
			\Delta E &= Q-W \\
			\Delta U &= Q-W
		\end{empheq}
		
		$\Delta U = \Delta E$ nur wenn $\Delta KE$ und $\Delta PE$ vernachlässigbar sind!
		
		Falls $Q = 0$ $\Rightarrow$ \emph{adiabater} Prozess $\Rightarrow$ $\Delta E = -W$
		
		Prozess $1 \longrightarrow 2$:
		\[
			\begin{array}{r@{\:}c@{\:}l@{\quad}l}
				Q_1 & \xrightarrow{\Delta Q} & Q_2 & \text{$\Delta Q$ \emph{Weg abhängig!}} \\ & & & \Rightarrow \text{ keine Zustandsgrösse} %\\
				% W_1 & \xrightarrow{\Delta W} & W_2 & \text{$\Delta W$ Weg unabhängig}
			\end{array}
		\]
		
	% subsection: 1. Hauptsatz (end)
	
	\subsection{Wärmeübertragung} % (fold)
		\begin{tabular*}{\textwidth}{@{}r@{\hspace{2mm}}l@{}}
			\textbf{Wärmeübertragungs-} & \\ \textbf{leistung:} & $\dot{Q} \nicesunit{\joule\per\second} = \unit{\watt}$ \\
			\textbf{Wärmefluss:} & $\dot{Q}''$ oder $\dot{q} \nicesunit{\joule\per\second\per\Square\metre} = \niceunit{\watt\per\Square\metre}$
		\end{tabular*}
	% subsection: Wärmeübertragung (end)
	
	\subsection{Die Leistungsbilanz} % (fold)
		\[
			\frac{\diff E}{\diff t} = \dot{E} = \dot{Q} - \dot{W}
		\]
		
		stationärer Zustand $\Leftrightarrow$ keine Zeitabhängigkeit $\Leftrightarrow$ $\dot{E} = 0$
		\[
			\Rightarrow \ \dot{W} = \dot{Q}
		\]
	% subsection: Die Leistungsbilanz (end)
	
	\subsection{Kreisprozesse} % (fold)
		\[
			\Delta E_{\text{KP}} = 0 \quad \Rightarrow \quad W_{\text{KP}} = Q_{\text{KP}}
		\]
		
		\subsubsection{Thermische Wirkungsgrade} % (fold)
			
			\paragraph{Bei thermischen Arbeitsmaschinen:} % (fold)
				\[
					W_{\text{KP}} = Q_{\text{zu}} - Q_{\text{ab}}
				\]
				
				\[
					\eta_{\text{th}} = \frac{W_{\text{KP}}}{Q_{\text{zu}}} = 1 - \frac{Q_{\text{ab}}}{Q_{\text{zu}}}
					= \frac{Q_{\text{zu}} - Q_{\text{ab}}}{Q_{\text{zu}}}
				\]
				
				$\eta_{\text{th}}$ theoretisch maximal $1$, da $Q_{\text{ab}}$ immer $> 0$ ist, muss $\eta_{\text{th}} < 1$ sein.
			% paragraph: Bei thermischen Arbeitsmaschinen: (end)
			
			\paragraph{Bei Kältemaschinen und Wärmepumpen:} % (fold)
				\begin{align*}
					-W_{\text{KP}} &= Q_{\text{zu}} - Q_{\text{ab}} \\
					\Rightarrow W_{\text{KP}} &= Q_{\text{ab}} - Q_{\text{zu}}
				\end{align*}
				
				\begin{align*}
					\varepsilon_{\text{K}} &= \frac{Q_{\text{zu}}}{W_{\text{KP}}} = \frac{Q_{\text{zu}}}{Q_{\text{ab}} - Q_{\text{zu}}} \\
					\varepsilon_{\text{W}} &= \frac{Q_{\text{ab}}}{W_{\text{KP}}} = \frac{Q_{\text{ab}}}{Q_{\text{ab}} - Q_{\text{zu}}} = \varepsilon_{\text{K}}+1
				\end{align*}
			% paragraph: Bei Kältemaschinen und Wärmepumpen: (end)
			
		% subsubsection: Thermische Wirkungsgrade (end)
	% subsection: Kreisprozesse (end)
% section: Erste Hauptsatz (end)