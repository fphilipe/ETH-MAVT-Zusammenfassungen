%!TEX root = ../Numerische Mathematik.tex

\section{Spektralradius}
	\begin{lstlisting}[style=list]
D = diag(diag(M));
R = triu(M) - D;
L = tril(M) - D;
T = D\(-L-R);
max(abs(eig(T)));
	\end{lstlisting}

\section{Jacobi-Verfahren} 
	\begin{lstlisting}[style=list]
while ( iter < maxIter )
	iter = iter + 1;
	x_old = x;
	x = D\((-L-R)*x + b); % Jacobi iteration without inverse
	if ( norm(x-x_old)/norm(x) < tol ) break; end % converged
end
	\end{lstlisting}

\section{Newton-Verfahren} 
	\begin{lstlisting}[style=list]
function[x_new, iter] = my_newton(x0,tol,maxit)
x_old = x0;
for iter=1:maxit
	Delta = -Df(x_old)\f(x_old);
	x_new = x_old + Delta;
	if ( norm(x_new-x_old) < tol*norm(x_new) )
		return
	end
	x_old = x_new
end
	\end{lstlisting}
	
\section{Quasi-Newton-Verfahren} 
	\begin{lstlisting}[style=list]
function[x_new,iter] = my_quasi_newton(x0,tol,maxit)
x_old = x0;
J = Df(x_old);
for iter=1:maxit
	Delta = -J\f(x_old);
	x_new = x_old + Delta;
	if ( norm(x_new-x_old) < tol*norm(x_new) )
		return;
	end
	x_old = x_new;
end
	\end{lstlisting}

\section{Singulärwertzerlegung (SVD)}
	\begin{lstlisting}[style=list]
B = [4 2 0; 3 5 7; -2 -1 0; 1 0 -1];
c = [10 13 -5 1.8]';
[u,s,v] = svd(B);
k = rank(B);
y = zeros(k,1);
for i = 1:k
	y(i) = u(:,i)' * c / s(i,i);
end
xstar = v(:,1:k) * y
	\end{lstlisting}[style=list]

\section{Fourier-Transform} 
	\begin{lstlisting}[style=list]
% Plotten der diskreten Funktionswerte
plot(x_k,f_k,'*');
hold on
% Berechnen der komplexen Koeffizienten
c = fft(f_k/N);

% Einfuegen von M-N Nullen
c_star = [c(1:N/2), zeros(1,M-N), c(N/2+1:end)];
% Auswertung des trigonometrischen Polynoms auf feinem Gitter
x_star = 2*pi/M*[0:M-1];
f_star = M*ifft(c_star);
	\end{lstlisting}

\section{Runge-Kutta-Verfahren -- Butcher-Tableau} 
	\begin{lstlisting}[style=list]
A = [0 0 0; .5 0 0; -1 2 0];
c = [0 .5 1]';
b = 1/6*[1 4 1]';
s = length(c); % Anzahl Stufen des RK Verfahrens
K = zeros(n,s); % Matrix mit K = [k_1,...,k_s]

% k Schritte
for j=1:k
	x_alt = x_hist(:,j);
	% s Stufen
	for i=1:s
		% Achtung: so nur falls Verfahren explizit, sonst Gl.sys. nach K(:,i) loesen.
		K(:,i) = f(t+c(i)*h, x_alt+h*K*A(i,:)');
	end
	x_hist(:,j+1) = x_alt+h*K*b;
	t = t+h;
end
	\end{lstlisting}

\section{Implizite Mittelpunktsregel}
	\begin{lstlisting}[style=list]
function[y_neu] = impl_mp_schritt(fct,t,y,h)
% Parameter fuer Fixpunktiter.
tol = 1e-6;
maxit = 100;
% Startwert fuer Fixpunktiteration
y_neu = y+h*fct(t,y);
y_vorh_schritt = y_neu;
for i=1:maxit
	y_neu = y+h*fct(t+h/2,(y+y_neu)/2);
	% Abbruchkriterium
	if ( norm(y_neu-y_vorh_schritt) < tol+tol*norm(y_neu) )
		return
	end
	y_vorh_schritt = y_neu;
end
disp('Fixpunktiteration hat nicht konvergiert')
	\end{lstlisting}

\section{Verfahren von Heun}
	\begin{lstlisting}[style=list]
function[t,y] = simple_heun(f,t0,y0,h,T)
%---------------------------------------
% Verfahren von Heun zum Loesen der ODE
%    y'=f(t,y), y(t0)=y0
%  f: Funktionshandle auf f(t,y)
% y0: Anfangswerte
% t0: Anfangszeit
%  h: Schrittweite
%  T: Endzeitpunkt
%  t: Zeitpunkte t_k in [t0, T]
%  y: Approximationen zu Zeitpunkten t
%---------------------------------------
t = t0;
y = y0;
ct = 1;
while ( t(ct) < T )
	f_ = f(t(ct), y(:,ct));
	y_temp = y(:,ct) + h*f_;
	y(:ct+1) = y(:,ct) + h/2*(f_ + f(t(ct) + h, y_temp));
	t(ct+1) = t(ct) + h;
	ct = ct+1;
end
	\end{lstlisting}
