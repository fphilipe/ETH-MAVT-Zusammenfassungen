%!TEX root = ../Analysis III.tex

\section{Laplace-Transformation} % (fold)
	Methode, um lineare Differentialgleichungen zu lösen.

	\begin{definition}
		Laplace-Transformierte $F(s)$ einer für $t > 0$ definierten Funktion $f(t)$:
		\emphequation{equation}{
			F(s) = \int_{0}^{\infty} f(t) \eu^{-st} \diff t \ . \label{laplace}
		}
	\end{definition}

	\begin{notation}
		$
			F(s) = \laplace{f}
		$
	\end{notation}

	\begin{bemerkung}
		$F(s)$ ist für jene $s$ definiert, für die das Integral in~\eqref{laplace} existiert.
	\end{bemerkung}
	
	\subsection{Eigenschaften} % (fold)
		\begin{description}
			\item[Linearität:]
			$
				\laplace{a f + b g} = a \cdot F(s) + b \cdot G(s)
			$ \\[1ex]
			wobei $f$, $g$ Funktionen und $a$, $b$ Konstanten sind.

			\item[Verschiebungssatz:]
			$
				\laplace{\eu^{-at}f}(s) = F(s+a)
			$

			\item[Ableitungsregel:]
			\begin{align*}
				\laplace{\frac{\diff f}{\diff t}} &= s \cdot F(s) - f(0) \\
				\laplace{\frac{\diff^2 f}{\diff t^2}} &= s^2 \cdot F(s) - s \cdot f(0) - f'(0) \\
				\laplace{\frac{\diff^n f}{\diff t^n}} &= s^n \cdot F(s) - \sum_{j=0}^{n-1} s^{n-1-j} \cdot  \frac{\diff^j f}{\diff t^j} \bigg|_0
			\end{align*}
			
			\item[Skalierungssatz:] $\laplace{f(at)} = \frac{1}{a} F \left(\frac{s}{a}\right)$
			
			\item[Faltung:] $\laplace{f*g} = \laplace f \cdot \laplace g$
		\end{description}

		\begin{center}
			\begin{tabular}{l@{$\qquad$}l}
				\toprule
				Zeitraum & Frequenzraum \\
				\midrule
				$\displaystyle
				(-t)^n f(t)\,,\ n\in\N
				$ & $\displaystyle
				\frac{\diff^n}{\diff s^n} F(s)
				$ \\
				$\displaystyle
				\int_0^t f(u) \diff u
				$ & $\displaystyle
				\frac{1}{s} F(s)
				$ \\
				$\displaystyle
				\frac{1}{t} f(t)
				$ & $\displaystyle
				\int_s^\infty F(u) \diff u
				$ \\
				$\displaystyle
				f(t-a) H(t-a)
				$ & $\displaystyle
				e^{-as} F(s)
				$ \\
				\bottomrule
			\end{tabular}
		\end{center}
	% subsection: Eigenschaften (end)
	
	\subsection{Korrespondenztabelle} % (fold)
		\begin{center}
			\begin{tabular}{l@{$\qquad$}l}
				\toprule
				Zeitraum & Frequenzraum \\
				\midrule
				$\displaystyle
				\delta(t-a)\,,\ a\geq 0
				$ & $\displaystyle
				\eu^{-as}
				$ \\
				$\displaystyle
				 H(t - a)\,,\ a\geq 0 
				$ & $\displaystyle
				 \frac{\eu^{-as}}{s} 
				$ \\
				$\displaystyle
				 t^n\,,\ n\in\N_0 
				$ & $\displaystyle
				 \frac{n!}{s^{n+1}} 
				$ \\
				$\displaystyle
				 \eu^{-at} 
				$ & $\displaystyle
				 \frac{1}{s+a} 
				$ \\
				$\displaystyle
				1- \eu^{-at}
				$ & $\displaystyle
				 \frac{a}{s(s+a)} 
				$ \\
				$\displaystyle
				\sin(at)
				$ & $\displaystyle
				\frac{a}{s^2+a^2}
				$ \\
				$\displaystyle
				\cos(at)
				$ & $\displaystyle
				\frac{s}{s^2+a^2}
				$ \\
				$\displaystyle
				\sinh(at)
				$ & $\displaystyle
				\frac{a}{s^2-a^2}
				$ \\
				$\displaystyle
				\cosh(at)
				$ & $\displaystyle
				\frac{s}{s^2-a^2}
				$ \\
				\bottomrule
			\end{tabular}
		\end{center}
	% subsection: Tabelle (end)

	\subsection{Die Inverse Laplace-Transformation} % (fold)
		\begin{notation}
			$
				f = \invlaplace{F}
			$
		\end{notation}

		\begin{bemerkungen}
			\item Es gibt Funktionen, die keine Inverse haben.
			\item Es gibt eine Umkehrformel, die in $\C$ operiert:\[
				f(t) = \frac{1}{2\pi\iu} \int_\Gamma \eu^{st} F(s) \diff s
			\] über den Integrationsweg $\Gamma$ im Komplexen.
			\item In den Fällen wo die Laplace-Transformation nützlich ist, findet
			man die Inverse durch Zurückführen auf Funktionen von denen die Inverse
			bekannt ist.
		\end{bemerkungen}
		
		\paragraph{Vorgehen bei der Rücktransformation} % (fold)
			% TODO: Vorgehen bei Rücktransformation verbessern
			\begin{tightitemize}
				\item Suche nach Verschiebungssatz
				\item Zerlege Faktoren durch Partialbruchzerlegung
				\item Quadratisch ergänzen, so dass Therme geklammert werden können
				\item Bei Rücktransformation mit dem Verschiebungssatz muss oft zwischen $t<a$ und $t>a$ unterschieden werden.
			\end{tightitemize}
		% paragraph: Vorgehen bei der Rücktransformation (end)
	% subsection: Die Inverse (end)
% section: Laplace-Transformation (end)