%!TEX root = ../Physik II.tex

\section{Konstanten} % (fold)
	{
		\setlength{\mathindent}{0pt}
		\begin{equation*}
			\begin{array}{@{}l@{\quad}r@{\:=\:}l}
				\text{Lichtgeschwindigkeit:} & c & \SI{3.00e8}{\metre\per\second} \\
				\text{Influenzkonstante:} & \epsilon_0 & \frac{10^7}{4\pi c^2} = \SI{8.85e-12}{\ampere\second\per\volt\per\metre} \\
				\text{Induktionskonstante:} & \mu_0 & \num[numaddn=\pi,mathsrm=mathnormal]{4\pi e-7}\SI{}{\volt\second\per\ampere\per\metre} \\
				\text{Elementarladung:} & e & \SI{1.60e-19}{\ampere\second} \\
				\text{Bolzmannkonstante:} & k_B & \SI{1.38e-23}{\joule\per\kelvin} \\
				\text{Avogadrozahl:} & N_A & \SI{6.02e23}{\per\mole} \\
				\text{Gaskonstante:} & R & k_B N_A = \SI{8.31}{\joule\per\kelvin\per\mole} \\
				\text{Erdbeschleunigung:} & g & \SI{9.81}{\metre\per\Square\second} \\
				\text{Plank'sches Wirkungsquantum:} & h & \SI{6.62e-34}{\joule\sec} \\
				\text{Masse eines Elektrons:} & m_e & \SI{9.11e-31}{\kilo\gram}
			\end{array}
		\end{equation*}
	}
% section: Konstanten (end)

\section{Mathematische Grundlagen} % (fold)
	
	\subsection{Der Feldbegriff} % (fold)
		
		\begin{description}
			
			\item[Skalarfeld]
			weist jedem Punkt $(x,y,z)$ eine skalare Grösse (zur Zeit $t$) zu.
			
			\item[Vektorfeld]
			weist jedem Punkt $(x,y,z)$ einen Vektor (zur Zeit $t$) zu.
			
		\end{description}
		
	% subsection: Der Feldbegriff (end)
	
	\subsection{Einführung in die Vektoranalysis} % (fold)
		
		\subsubsection{Differentiation von skalaren Feldern -- der Gradient} % (fold)
			
			\begin{definition}
				\textbf{Gradient} $\grad$ vom skalaren Feld $T(x,y,z)$ ist
				\[
					\gradient T = \grad T = (\grad_x T, \grad_y T, \grad_z T) = \left( \frac{\partial T}{\partial x}, \frac{\partial T}{\partial y}, \frac{\partial T}{\partial z} \right)
				\]
			\end{definition}
			
			Die Differenz von zwei Punkten eines Skalarfeldes ist
			\[
				\Delta T = \grad T \cdot \overrightarrow{P_1 P_2}
			\]
			
			\begin{satz}
				Der Gradient eines Feldes $F(x,y,z)$ steht senkrecht auf den Flächen $F(x,y,z) = \const$.
			\end{satz}
			
		% subsubsection: Differentiation von skalaren Feldern -- der Gradient (end)
		
		\subsubsection{Der Operator $\grad$} % (fold)
			
			\paragraph{Vektoroperator} % (fold)
				\[
					\grad = \left( \frac{\partial}{\partial x}, \frac{\partial}{\partial y}, \frac{\partial}{\partial z} \right)
				\]
			% paragraph: Vektoroperator (end)
			
			\[
				\begin{array}{r@{\ =\ }l@{\ \rightarrow \ }l}
					\grad T & \gradient T & \text{Vektor} \\
					\Div \vec{h} & \divergenz \vec{h} & \text{Skalar} \\
					\rot \vec{h} & \rotation \vec{h} & \text{Vektor}
				\end{array}
			\]
			
			\paragraph{Beziehungen} % (fold)
				\begin{align*}
					\rotation(\gradient T) &= \rot (\grad T) = 0 \\
					\divergenz(\rotation \vec{h}) &= \Div (\rot \vec{h}) = 0
				\end{align*}
				
				\begin{theorem}
					Wenn die Rotation eines Vektorfeldes $\vec{A}$ verschwindet ($\rot \vec{A} = 0$), dann ist $\vec{A}$ immer der Gradient eines skalaren Feldes $\phi$, so dass $\vec{A} = \grad \phi$.
				\end{theorem}
				
				\begin{theorem}
					Wenn die Divergenz eines Vektorfeldes verschwindet ($\Div \vec{D} = 0$), dann existiert immer ein Vektorfeld $\vec{C}$, so dass gilt $\vec{D} = \rot \vec{C}$.
				\end{theorem}
				
				\subparagraph{Laplace-Operator} % (fold)
					
					\[
						\grad^2 T = T_{xx} + T_{yy} + T_{zz}
					\]
					
					Kann auch auf ein Vektorfeld angewendet werden, wobei sich wieder ein Vektorfeld ergibt.
					
				% subparagraph: Laplace-Operator (end)
			% paragraph: Beziehungen (end)
			
		% subsubsection: Der Operator $\grad$ (end)
		
	% subsection: Einführung in die Vektoranalysis (end)
	
	\subsection{Wichtige Vektorintegrale} % (fold)
		
		\subsubsection{Das Linienintegral von $\grad \phi$} % (fold)
			
			\[
				\phi(2) - \phi(1) = \int_{\mathrlap{
					\footnotesize{\begin{array}{@{}l@{}}
						(1) \\[-1ex]
						\text{längs } \Gamma
					\end{array}}}
				}^{(2)} (\grad \phi) \cdot \diff \vec{s}
			\]
			
		% subsubsection: Das Linienintegral von $\grad \phi$ (end)
		
		\subsubsection{Die Zirkulation eines Vektorfeldes} % (fold)
			
			Das Integral eines beliebigen Vektorfeldes $\vec{C}$ um einen \emph{geschlossenen} Weg $\Gamma$ nennt man Zirkulation des Vektorfeldes $\vec{C}$ um $\Gamma$:
			\[
				\oint_{\Gamma} \vec{C} \cdot \diff \vec{s} = \oint_{\Gamma} C_t \diff s
			\]
			wobei $C_t$ die Tangentialkomponente von $C$ ist.
			
		% subsubsection: Die Zirkulation eines Vektorfeldes (end)
		
		\subsubsection{Der Satz von Stokes} % (fold)
			
			\[
				\oint_{\Gamma} \vec{C} \cdot \diff \vec{s} = \int_S (\rot \vec{C} ) \cdot \vec{n} \diff f = \int_S (\rot \vec{C} ) \cdot \diff \vec{f}
			\]
			wobei $S$ \emph{irgendeine} von $\Gamma$ berandete Fläche ist und $\vec{n}$ ein Vektor, der senkrecht auf dem Flächenelement $\diff f$ steht und die Länge $\diff f$ hat.
			
		% subsubsection: Der Satz von Stokes (end)
		
		\subsubsection{Der Fluss eines Vektorfeldes} % (fold)
			
			Fluss von $\vec{h}$ durch die Fläche $S$:
			\[
				\int_S \vec{h} \cdot \diff \vec{f}
			\]
			wobei $\diff \vec{f}$ vom Volumen heraus zeigt.
			
		% subsubsection: Der Fluss eines Vektorfeldes (end)
		
		\subsubsection{Der Satz von Gauss} % (fold)
			
			\[
				\int_S \vec C \cdot \diff \vec f = \int_V \Div \vec C \diff V
			\]
			wobei $S$ \emph{irgendeine} geschlossene Fläche und $V$ das von ihr umschlossene Volumen ist.
			
		% subsubsection: Der Satz von Gauss (end)
		
	% subsection: Wichtige Vektorintegrale (end)
	
	\subsection{Fourier-Transformation} % (fold)
		\paragraph{Orts- $\rightarrow$ Impulsraum} % (fold)
			\[
				A(\omega) = \sqrt{\frac{1}{2\pi}} \int_{-\infty}^{\infty} f(y) \, \eu^{-\iu \omega y} \diff y
			\]
		% paragraph orts_rightarrow_impulsraum (end)
		\paragraph{Impuls- $\rightarrow$ Ortsraum} % (fold)
			\[
				f(y) = \sqrt{\frac{1}{2\pi}} \int_{-\infty}^{\infty} A(k_y) \, \eu^{\iu k_y y} \diff k_y
			\]
		% paragraph impuls_rightarrow_ortsraum (end)
	% subsection fourier_transformation (end)

	\subsubsection{Mathematische Identitäten und Näherungen} % (fold)
		\begin{align*}
			\sin x &= \frac{1}{2\iu} \parens{\eu^{\iu x} - \eu^{-\iu x}} \\
			\cos x &= \frac{1}{2} \parens{\eu^{\iu x} + \eu^{-\iu x}} \\
			\sin^2 x &= \half \parens{1 - \cos 2x} \\
			\cos^2 x &= \half \parens{1 + \cos 2x} \\
			\sinc x &= \frac{\sin x}{x} \\
			\sqrt{1 + x} &= 1 + \half x + \mathcal O (x^2) \\
			\eu^x &= 1 + x + \mathcal O (x^2) \\
			\sum_{n=0}^{N-1} x^n &= \begin{dcases}
				\frac{1-x^N}{1-x} & \text{falls } x \neq 1 \\
				N & \text{falls } x = 1
			\end{dcases}
		\end{align*}
	% subsubsection Identitäten und Näherungen (end)

	\subsubsection{Lagrange} % (fold)
		\[
			\Diff{}{t} \parens{\Part{T}{\dot{\vec q}}}^\transp
				- \parens{\Part{T}{\vec q}}^\transp + \parens{\Part{V}{\vec q}}^\transp
				= 0
		\]

		Wobei
		\begin{align*}
			V &= mgz \\
			V_\text{Feder} &= \half c (z - z_0)^2 \\
			T &= \half m \dot{\vec r}^2 + \half \Theta \dot \phi^2 \\
			T_\text{Masse an Seil} &= \half m l^2 \dot \phi^2
		\end{align*}
	% subsubsection Lagrange (end)

	\subsubsection{Elektrotechnik} % (fold)
		\begin{alignat*}{2}
			v_C &= \frac{1}{C} \int i_C \diff t & \qquad
			i_C &= C \Diff{v_C}{t} \\
			v_L &= L \Diff{i_L}{t} &
			i_L &= \frac{1}{L} \int v_L \diff t
		\end{alignat*}
	% subsubsection Elektrotechnik (end)
	
% section: Mathematische Grundlagen (end)
