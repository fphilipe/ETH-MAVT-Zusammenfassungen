%!TEX root = /Users/philipe/Documents/ETH/3. Semester/Elektrotechnik/Zusammenfassung/Neu/Elektrotechnik Zusammenfassung.tex

\begin{circuitikz}[scale=1.5,>=latex]
	\usetikzlibrary{calc}
	\begin{scope}
		\draw
			(opamp) node[op amp] {}
			(opamp.+) node[above] {$v_+$}
			(opamp.-) node[below] {$v_-$}
			(opamp.out) node{}
		;
		\coordinate (ground1) at ($(opamp.-) + (-1.5,-1.5)$);
			
		\draw
			(ground1) node[ground]{}
			-- ($(opamp.-)+(-1.5,0)$)
			to[R=$R$, -*] (opamp.-)
		;
		
		\draw
			(opamp.-) -- ++(0,.5)
			to[R=$(K-1)R$] ($(opamp.out)+(0,.825)$)
			-- (opamp.out)
		;
		
		\draw
			(ground1) ++(1,0)
			node[ground]{}
			to[V_=$v_{\text{i}+}$] ($(opamp.+)+(-.5,0)$)
			-- (opamp.+)
		;
		
		\draw
			(opamp.out) ++(.5,0) to[short, o-*] (opamp.out)
			(opamp.out) ++(.5,-.5) node[ground]{}
			to[open, o-] (opamp.out)
		;
		
		\draw[->]
			(opamp.out) ++(.625,0) arc (45:-45:10pt)
		;
		\draw
			(opamp.out)++(.75,-.25) node[left]{$v_{\text o}$}
		;
	\end{scope}
	
	\begin{scope}[xshift=2.5cm,scale=.75,yshift=-1cm]
		\draw
			(0,0) node[ground]{}
			to[V_=$v_{\text{i}+}$] (0,1)
			-- (1,1)
			(1,.33) -- (1,1.66) -- (2,1) -- cycle
			(1.5,1) node{K}
			(2,1) to[short, -o] (3,1)
			to[open, -o] (3,.25)
			-- (3,0)
			node[ground]{}
		;
		\draw[->]
			(3.125,1) arc (45:-45:15pt)
		;
		\draw
			(3.25,.625) node[left]{$v_{\text o}$}
		;
	\end{scope}
\end{circuitikz}