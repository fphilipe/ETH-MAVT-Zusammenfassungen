%!TEX root = ../Elektrotechnik I.tex

\begin{circuitikz}[scale=1.5,>=latex]
	\usetikzlibrary{calc}
	\draw
		(opamp) node[op amp] {}
		(opamp.+) node[above] {$v_+$}
		(opamp.-) node[below] {$v_-$}
		(opamp.out) node{}
	;
	\coordinate (ground1) at ($(opamp.-) + (-1.5,-1.5)$);
		
	\draw
		(ground1) node[ground]{}
		to[V_=$v_{\text{i}-}$] ($(opamp.-)+(-1.5,0)$)
		to[R=$Z_1$, -*] (opamp.-)
	;
	
	\draw
		(opamp.-) -- ++(0,.5)
		to[R=$Z_2$] ($(opamp.out)+(0,.825)$)
		-- (opamp.out)
	;
	
	\draw
		(ground1) ++(1,0)
		node[ground]{}
		to[V_=$v_{\text{i}+}$] ($(opamp.+)+(-.5,0)$)
		-- (opamp.+)
	;
	
	\draw
		(opamp.out) ++(.5,0) to[short, o-*] (opamp.out)
		(opamp.out) ++(.5,-.5) node[ground]{}
		to[open, o-] (opamp.out)
	;
	
	\draw[->]
		(opamp.out) ++(.625,0) arc (45:-45:10pt)
	;
	\draw
		(opamp.out)++(.75,-.25) node[left]{$v_{\text o}$}
	;
\end{circuitikz}